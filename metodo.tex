\chapter[Materiais e métodos]{Materiais e métodos}

\section{Algoritmo}

O algoritmo proposto é baseado no método de seleção de variáveis \textit{stepwise}, comumente utilizado em regressão linear. 
Nele cada variável é ajustada isoladamente à saída e a performance dos modelos são aferidas; seleciona-se a variável que apresentar
o melhor resultado. 

Ajustam-se novamente modelos considerando-se cada uma das variáveis restantes em conjunto com as previamente selecionadas; a 
variável correspondente ao modelo de melhor performance é adicionada ao conjunto de variáveis selecionadas. Repete-se o procedimento
até que não haja mais melhora de performance nos modelos treinados, ou esgotem-se as variáveis.

\begin{algorithm}
    \caption{\textit{Forward Stepwise Selection}}
    \begin{algorithmic}
        \Function{$forwardSelection$}{$variáveis,saídas,limiar$}
            \State $selecionadas \gets NULL$
            \Repeat
            \State $melhorErro \gets \infty$
            \ForAll{$variáveis$}
            
            \EndFor
            \Until{$variáveis == NULL ou erro<limiar$}
        \EndFunction
    \end{algorithmic}
\end{algorithm}
